This paper presets a finite element approach to solving reaction-diffusion PDEs. I discretize the spacial derivative for the diffusion term according to the finite element method. Then, I combine the implicit and explicit Euler methods to calculate numerically stable time steps given the nonlinear reaction. Because I use the FEM, I can solve reaction-diffusion systems on a wide range of domains, including ones with complex structure and on 3D surfaces.

Across all surfaces, the reaction-diffusion equation develops similarly from random noise, to diffuse hotspots, and finally to well-defined Turing patterns. The Turing patterns that develop depend on the parameter choices within the PDE, but the general qualitative properties, including the shape and width, of the patterns are robust across square and circle domains. Exploration into the numerical properties of the method suggest it generally gets a reasonable degree of accuracy, but can have some error, especially on the less smooth parts of the function.

To expand on this method, future projects should search for more accurate solutions with more realistic applications. One approach would be to implement the spectral element method \parencite{hafeez2023review} and use polynomials in place of my linear $\psi_i$ functions. Alternatively, on 3D domains the curvature of the domain can affect the solution to the PDE \parencites{staddon2024zebra}{leon2021full}, so future work could look into the effect of this curvature. Specific to animal pigmentation patterns, a more realistic model could incorporate the effect of tissue growth in reaction-diffusion systems \parencite{de2020leopard}.
