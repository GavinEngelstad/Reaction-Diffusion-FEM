Reaction-diffusion systems model the spatiotemporal behavior of chemical and physical systems of one of more components that react with each other and diffuse across space. Chemical reactions \parencites{zhabotinsky2007belousov}{graham1993temperature}, the human nervous system \parencite{fitzhugh1961impulses}, population dynamics \parencite{clair2024reaction}, and the patterns that show up on animal's skins \parencites{turing1990chemical}{de2020leopard} can all be described using versions of a reaction-diffusion system. Within a system, reactions turn one substance into another as diffusion causes substances to spread out \parencite{li2020reaction}. The solutions to reaction-diffusion systems presents interesting patterns (dubbed ``Turing Patterns'' \parencite{vittadello2021turing}), moving fronts, and oscillations \parencites{szalai2004turing}{rinzel1982propagation}{turing1990chemical}.

Because of the wide range of applications for reaction-diffusion systems, it is important to have accurate and efficient computational methods to solve them. The main challenge to solving the reaction-diffusion systems is the nonlinear reaction term \parencites{pao1982nonlinear}{martin1992nonlinear}. This term drives the asymptotic behavior and stability of the system \parencite{pao1982nonlinear}, but it also introduces additional complexity, especially for methods that abuse local, linear approximations of the system to solve it. Finite differences \parencite{hoff1978stability}, spectral \parencites{bueno2014fourier}{craster2018spectral}, and analytic \parencites{spendier2013analytic} approaches can solve this problem with various degrees of computational efficiency and accuracy, but are limited to only work for certain reaction-diffusion systems or domain shapes.

This paper presents an approach to solving reaction-diffusion systems that uses the finite element method (FEM). The method in txhis paper is heavily based on the mathematical approach of \autocite{sellami2020accelerating} and \autocite{lang1992finite}. I utilize finite elements to discretize the space derivative and a combined implicit (backwards) and explicit (forwards) Euler method to discretize the time derivative. Using a fine enough triangular mesh, the method is numerically stable and can be applied to a wide range of domains, including standard disks and rectangles, more complicated maze-like structures, and along the surface of 3D shapes.
